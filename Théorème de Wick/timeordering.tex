
\documentclass[12pt]{article}
\usepackage[T1]{fontenc}
\usepackage[utf8]{inputenc}
\usepackage{amsmath,amssymb,amsthm,mathtools}
\usepackage{braket} 
\usepackage{physics} 
\usepackage{tikz}
\usepackage{dsfont}
\usetikzlibrary{positioning}
\usepackage{etoolbox}
\usepackage{simpler-wick} %POUR LES CONTRACTIONS DE WICK

\title{DEVOIR QFT: Théorème de Wick}
\author{
    % D'abord vos informations personnelles
    RAZAFIMAHERY Faneva Liantsoa \\
    \medskip % Un petit espace
    \normalsize Master 1 -Physique des Hautes Energies \\
    \medskip
    \normalsize Université d'Antananarivo  \\
    %\vspace{1cm} %  pour séparer verticalement bro
    % Ensuite, on insère le titre manuellement
    %\Large \textbf{DEVOIR QFT: Théorème de Wick} \\ si on veut mettre le titre ici et en gras
    % \Huge C'est grand et \LARGE ou \Large c plus petit
}
\date{6 Novembre 2025} 


\begin{document}
\maketitle

\underline{Question}: Montrer que l'on a:



%\subsection*{Placeholders for contractions}


\begin{equation*}
\begin{split}
\mathcal{T}\{\phi(x_1)\phi(x_2)\phi(x_3)\phi(x_4)\}
&=\mathcal{T}\{\phi(x_1)\phi(x_2)\}\mathcal{T}\{\phi(x_3)\phi(x_4)\}
+\mathcal{T}\{\phi(x_1)\phi(x_2)\}\mathcal{T}\{\phi(x_3)\phi(x_4)\}\\
&+\mathcal{T}\{\phi(x_1)\phi(x_2)\}\mathcal{T}\{\phi(x_3)\phi(x_4)\}
\end{split}
 \end{equation*}

\underline{Réponse}:
On connait le théorème de Wick qui dit que:
\begin{equation*}
\begin{split}
\mathcal{T}\{\prod_{k=1}^{m}\phi(x_{k})\}&= N\{\prod_{k=1}^{m}\phi(x_{k})\}+
\sum_{\alpha,\beta}\wick{\c\phi(x_{\alpha}) \c\phi(x_{\beta})}
N\{\prod_{k\neq\alpha,\beta}^{m}\phi(x_{k})\}\\
&+\sum_{(\alpha,\beta),(\gamma,\delta)}\wick{\c1\phi(x_{\alpha}) \c1\phi(x_{\beta})
\c2\phi(x_{\gamma}) \c2\phi(x_{\delta})}
N\{\prod_{k\neq\alpha,\beta,\gamma,\delta}^{m}\phi(x_{k})\}  \\
&+...
\end{split}
\end{equation*}

Simplifions notre écriture en faisant
\begin{equation*} 
\mathcal{T}\{\phi(x_1)\phi(x_2)\}=\mathcal{T}\{\phi_{1}\phi_{2}\}
\end{equation*}

Ainsi pour 4 opérateurs de champ, on a:
\begin{equation*}
\mathcal{T}\{\phi(x_1)\phi(x_2)\phi(x_3)\phi(x_4)\}
=\mathcal{T}\{\phi_{1}\phi_{2}\phi_{3}\phi_{4}\}
\end{equation*}

On peut donc écrire de la manière suivante grace au théorème de Wick:
\begin{equation}
\begin{split}
\mathcal{T}\{\phi_{1}\phi_{2}\phi_{3}\phi_{4}\}
&=N\{\phi_{1}\phi_{2}\phi_{3}\phi_{4}\}
 +N\{\wick{\c  \phi_{1} \c  \phi_{2}     \phi_{3}     \phi_{4}}\}
 +N\{\wick{\c  \phi_{1}     \phi_{2} \c  \phi_{3}     \phi_{4}}\}
 +N\{\wick{\c  \phi_{1}     \phi_{2}     \phi_{3} \c  \phi_{4}}\}\\
&+N\{\wick{    \phi_{1} \c  \phi_{2} \c  \phi_{3}     \phi_{4}}\}
 +N\{\wick{    \phi_{1} \c  \phi_{2}     \phi_{3} \c  \phi_{4}}\}
 +N\{\wick{    \phi_{1}     \phi_{2} \c  \phi_{3} \c  \phi_{4}}\}
 +N\{\wick{\c1 \phi_{1} \c1 \phi_{2} \c2 \phi_{3} \c2 \phi_{4}}\}\\
&+N\{\wick{\c1 \phi_{1} \c2 \phi_{2} \c1 \phi_{3} \c2 \phi_{4}}\}
 +N\{\wick{\c1 \phi_{1} \c2 \phi_{2} \c2 \phi_{3} \c1 \phi_{4}}\}
\end{split}
\end{equation}

On définit $N\{\wick{\c\phi_{1}\c\phi_{2}\phi_{3}\phi_{4}}\}$ de telle sorte que:
\begin{equation*}
N\{\wick{\c\phi_{1}\c\phi_{2}\phi_{3}\phi_{4}}\}=
\Delta_{F}(x_{1}-x_{2})\mathds{1}N\{\phi_{3}\phi_{4}\}
\end{equation*}

Donc on réécrit l'équation (1) comme suit:
\begin{equation}
\begin{split}
\mathcal{T}\{\phi_{1}\phi_{2}\phi_{3}\phi_{4}\}&= N\{\phi_{1}\phi_{2}\phi_{3}\phi_{4}\}
+\Delta_{F}(x_{1}-x_{2})\mathds{1}N\{\phi_{3}\phi_{4}\}
+\Delta_{F}(x_{1}-x_{3})\mathds{1}N\{\phi_{2}\phi_{4}\}\\
&+\Delta_{F}(x_{1}-x_{4})\mathds{1}N\{\phi_{2}\phi_{3}\}
+\Delta_{F}(x_{2}-x_{3})\mathds{1}N\{\phi_{1}\phi_{4}\}
+\Delta_{F}(x_{2}-x_{4})\mathds{1}N\{\phi_{1}\phi_{3}\}\\
&+\Delta_{F}(x_{3}-x_{4})\mathds{1}N\{\phi_{1}\phi_{2}\}
+\Delta_{F}(x_{1}-x_{2})\mathds{1}\Delta_{F}(x_{3}-x_{4})\mathds{1}\\
&+\Delta_{F}(x_{1}-x_{3})\mathds{1}\Delta_{F}(x_{2}-x_{4})\mathds{1}
+\Delta_{F}(x_{1}-x_{4})\mathds{1}\Delta_{F}(x_{2}-x_{3})\mathds{1}
\end{split}
\end{equation}

Or on sait que on peut poser:
\begin{align*}
\phi_{x}=\phi_{x}^{+}+\phi_{x}^{-}\\
\phi_{y}=\phi_{y}^{+}+\phi_{y}^{-}
\end{align*} 

Ce qui nous permet d'écrire d'après le cours:
\begin{equation}    
\begin{split}
\mathcal{T}\{\phi_{x}\phi_{y}\}=\Theta(x^{0}-y^{0})([\phi_{x}^{+},\phi_{y}^{-}] 
+N\{\phi_{x}\phi_{y}\})+\Theta(y^{0}-x^{0})([\phi_{y}^{+},\phi_{x}^{-}]
+N\{\phi_{y}\phi_{x}\})
\end{split}
\end{equation}

Aussi que:
\begin{equation}
\Delta_{F}(x-y)=\Theta(x^{0}-y^{0}) \braket{0}{[\phi_{x}^{+}\phi_{y}^{-}]|0}+
\Theta(y^{0}-x^{0}) \braket{0}{[\phi_{y}^{+}\phi_{x}^{-}]|0}
\end{equation}

On peut identifier les équations (3) et (4) pour obtenir:
\begin{equation}
\mathcal{T}\{\phi_{x}\phi_{y}\}= \Delta_{F}(x-y)\mathds{1}+N\{\phi_{x}\phi_{y}\}
+N\{\phi_{y}\phi_{x}\}
\end{equation}

Où on définit $N\{\phi_{y}\phi_{x}\}$ comme le normal ordering de $\phi_{y}\phi_{x}$



Ainsi chaque terme de l'équation (2) peut s'écrire comme suit:
\begin{align*}
\Delta_{F}(x_{1}-x_{2})\mathds{1}=
\mathcal{T}\{\phi_{1}\phi_{2}\}-\Theta(x_{1}-x_{2})N\{\phi_{1}\phi_{2}\}
-\Theta(x_{2}-x_{1})N\{\phi_{2}\phi_{1}\}\\
\Delta_{F}(x_{1}-x_{3})\mathds{1}=
\mathcal{T}\{\phi_{1}\phi_{3}\}-\Theta(x_{1}-x_{3})N\{\phi_{1}\phi_{3}\}
-\Theta(x_{3}-x_{1})N\{\phi_{3}\phi_{1}\}\\
\Delta_{F}(x_{1}-x_{4})\mathds{1}=
\mathcal{T}\{\phi_{1}\phi_{4}\}-\Theta(x_{1}-x_{4})N\{\phi_{1}\phi_{4}\}
-\Theta(x_{4}-x_{1})N\{\phi_{4}\phi_{1}\}\\
\Delta_{F}(x_{2}-x_{3})\mathds{1}=
\mathcal{T}\{\phi_{2}\phi_{3}\}-\Theta(x_{2}-x_{3})N\{\phi_{2}\phi_{3}\}
-\Theta(x_{3}-x_{2})N\{\phi_{3}\phi_{2}\}\\
\Delta_{F}(x_{2}-x_{4})\mathds{1}=
\mathcal{T}\{\phi_{2}\phi_{4}\}-\Theta(x_{2}-x_{4})N\{\phi_{2}\phi_{4}\}
-\Theta(x_{4}-x_{2})N\{\phi_{4}\phi_{2}\}\\
\Delta_{F}(x_{3}-x_{4})\mathds{1}=
\mathcal{T}\{\phi_{3}\phi_{4}\}-\Theta(x_{3}-x_{4})N\{\phi_{3}\phi_{4}\}
-\Theta(x_{4}-x_{3})N\{\phi_{4}\phi_{3}\}\\
\end{align*}

Prenons pour exemple le second terme du membre de droite de l'équation (2):
\begin{equation*}
\Delta_{F}(x_{1}-x_{2})\mathds{1}N\{\phi_{3}\phi_{4}\}=
(\mathcal{T}\{\phi_{1}\phi_{2}\}-\Theta(x_{1}-x_{2})N\{\phi_{1}\phi_{2}\}
-\Theta(x_{2}-x_{1})N\{\phi_{2}\phi_{1}\})(N\{\phi_{3}\phi_{4}\})
\end{equation*}

Or de par la définition du normal ordering, on sait que les termes contenant les normal ordering disparaissent sous $\braket{0}{.|0}$, donc on a:
\begin{equation}
    \braket{0}{\Delta_{F}(x_{1}-x_{2})\mathds{1}N\{\phi_{3}\phi_{4}\}|0}=0
\end{equation}

Donc si on effectue le calcul $\braket{0}{.\mathcal{T}\{\phi(x_1)\phi(x_2)\phi(x_3)\phi(x_4)\}|0}$ et en tenant compte de (6),on obtient:
\begin{equation}
\begin{split}
&\bra{0}\mathcal{T}\{\phi_{1}\phi_{2}\phi_{3}\phi_{4}\}\ket{0}\\
&=\bra{0}(\mathcal{T}\{\phi_{1}\phi_{2}\}-\Theta(x_{1}-x_{2})N\{\phi_{1}\phi_{2}\}
-\Theta(x_{2}-x_{1})N\{\phi_{2}\phi_{1}\})\times{}\\
&\quad(\mathcal{T}\{\phi_{3}\phi_{4}\}-\Theta(x_{3}-x_{4})N\{\phi_{3}\phi_{4}\}
-\Theta(x_{4}-x_{3})N\{\phi_{4}\phi_{3}\})\ket{0}\\
&+\bra{0}(\mathcal{T}\{\phi_{1}\phi_{3}\}-\Theta(x_{1}-x_{3})N\{\phi_{1}\phi_{3}\}
-\Theta(x_{3}-x_{1})N\{\phi_{3}\phi_{1}\})\times{}\\
&\quad(\mathcal{T}\{\phi_{2}\phi_{4}\}-\Theta(x_{2}-x_{4})N\{\phi_{2}\phi_{4}\}
-\Theta(x_{4}-x_{2})N\{\phi_{4}\phi_{2}\})\ket{0}\\
&+\bra{0}(\mathcal{T}\{\phi_{1}\phi_{4}\}-\Theta(x_{1}-x_{4})N\{\phi_{1}\phi_{4}\}
-\Theta(x_{4}-x_{1})N\{\phi_{4}\phi_{1}\})\times{}\\
&\quad(\mathcal{T}\{\phi_{2}\phi_{3}\}-\Theta(x_{2}-x_{3})N\{\phi_{2}\phi_{3}\}
-\Theta(x_{3}-x_{2})N\{\phi_{3}\phi_{2}\})\ket{0}
\end{split}
\end{equation}

En nous rappelant que les termes contenant les normal ordering disparaissent sous $\braket{0}{.|0}$, on obtient:
\begin{equation}
\begin{split}
&\bra{0}\mathcal{T}\{\phi_{1}\phi_{2}\phi_{3}\phi_{4}\}\ket{0}\\
&=\bra{0}\mathcal{T}\{\phi_{1}\phi_{2}\}\mathcal{T}\{\phi_{3}\phi_{4}\}\ket{0}\\
&+\bra{0}\mathcal{T}\{\phi_{1}\phi_{3}\}\mathcal{T}\{\phi_{2}\phi_{4}\}\ket{0}\\
&+\bra{0}\mathcal{T}\{\phi_{1}\phi_{4}\}\mathcal{T}\{\phi_{2}\phi_{3}\}\ket{0}
\end{split}
\end{equation}

On peut regrouper chaque terme sous $\braket{0}{.|0}$ pour obtenir:
\begin{equation}
\begin{split}
&\bra{0}\mathcal{T}\{\phi_{1}\phi_{2}\phi_{3}\phi_{4}\}\ket{0}\\
&=\bra{0}(\mathcal{T}\{\phi_{1}\phi_{2}\}\mathcal{T}\{\phi_{3}\phi_{4}\}
+\mathcal{T}\{\phi_{1}\phi_{3}\}\mathcal{T}\{\phi_{2}\phi_{4}\}
+\mathcal{T}\{\phi_{1}\phi_{4}\}\mathcal{T}\{\phi_{2}\phi_{3}\})\ket{0}
\end{split}
\end{equation}

Finalement, on obtient le résultat attendu qui est :
\begin{equation}
\mathcal{T}\{\phi(x_1)\phi(x_2)\phi(x_3)\phi(x_4)\}=
\mathcal{T}\{\phi_{1}\phi_{2}\}\mathcal{T}\{\phi_{3}\phi_{4}\}
+\mathcal{T}\{\phi_{1}\phi_{3}\}\mathcal{T}\{\phi_{2}\phi_{4}\}
+\mathcal{T}\{\phi_{1}\phi_{4}\}\mathcal{T}\{\phi_{2}\phi_{3}\}
\end{equation}  

\end{document}

